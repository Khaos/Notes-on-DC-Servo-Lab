%% LyX 1.6.9 created this file.  For more info, see http://www.lyx.org/.
%% Do not edit unless you really know what you are doing.
\documentclass[10pt,a4paper,british]{article}
\usepackage[]{fontenc}
\usepackage[latin9]{inputenc}
\usepackage{booktabs}

\makeatletter

%%%%%%%%%%%%%%%%%%%%%%%%%%%%%% LyX specific LaTeX commands.
\pdfpageheight\paperheight
\pdfpagewidth\paperwidth

%% Because html converters don't know tabularnewline
\providecommand{\tabularnewline}{\\}

%%%%%%%%%%%%%%%%%%%%%%%%%%%%%% Textclass specific LaTeX commands.
 \usepackage[headings]{DJespcrc1}

%%%%%%%%%%%%%%%%%%%%%%%%%%%%%% User specified LaTeX commands.
%\input{DJcommonpk-phd1-pgf-en}
%\input{DJcommonpk-pgf-en}

% Definition of command \djphd
\input{DJ4DJespcrc}

% Commonly used mathematical macro
%\input{DJcommonmath}
% identification
%\readRCS $Id: ModifiedFICA.tex,v 0.01 2009/06/15 15:20:11 Exp $
%\ProvidesFile{ModifiedFICA.tex}[\filedate \space v\fileversion
%     \space DJElsevier 1-column CRC Author Instructions]
\readRCS $Id: DJACS220Notes.tex,v 0.01 2010/02/20 15:20:11 Exp $
\ProvidesFile{DJACS220Notes.tex}[\filedate \space v\fileversion
     \space DJElsevier 1-column CRC Author Instructions]



% This package is not declared in DJespcrc1.sty
%\usepackage[numbers, sort&compress]{natbib}
% pdfsync.sty allows one to synchronize between LaTeX source and pdf output
%\usepackage{pdfsync}

%% !!! Comment
%%     The package 'xcolor' has to be loaded before the package 'pstricks'.
\usepackage[svgnames]{xcolor}



% change this to the following line for use with LaTeX2.09
% \documentstyle[12pt,twoside,fleqn,espcrc1]{article}
% if you want to include PostScript figures
%\usepackage{graphicx}
% if you have landscape tables
%\usepackage[figuresright]{rotating}
% Put all the style files and packages you want here
% Load DJ-defined report packages
\input{DJReportLoadAll}

%\usepackage{amsmath}

\usepackage{mathtools}

\usepackage{siunitx}

% \usepackage{djschemabloc}


\usetikzlibrary{circuits,%
    intersections,%
    scopes, arrows, fadings, patterns,%
    decorations.pathreplacing,%
    decorations.pathmorphing,%
    positioning,
    shadows,
}


\usepackage{pgfplots}
%% -*- Mode: LaTeX Memoir; tab-width: 4;

%% defpgfexternal.tex    Externalization library
%% Load the externalization library
%% MAIN FILE: Dummy.tex
%% Created by Dazhi Jiang, 2011-02-01 16:34:27 +0000 (Tue,  1 Feb 2011)
%% Copyright (c) 2011 Dazhi Jiang. All Rights Reserved. 

%% TextMate Settings
%!TEX root = Dummy.tex

\newcommand{\setpgfexternalmainfile}[1]{%
	\usepgfplotslibrary{external}
	\tikzexternalize[
		prefix=Fig/pdf/pgf-,
		export=false, % needs to be activated for single pictures (i.e. expensive ones)
		mode=list and make,
		verbose IO=false,
		%xport=true,% FASTER FOR DEBUGGING
	]{#1}
		%% !!!
		%% Changed by Dazhi Jiang - 2010-09-20
		%% The following line is commented since the new definition of
		%% \tikzexternalize in PGF cvs
		% {pgfplots}
	\tikzifexternalizing{%
		\nofiles
	}{}%
}%

\setpgfexternalmainfile{NotesonProjectsPartI}

\usepackage[style=altlong4colheader,toc]{glossaries}

\setlength{\glsdescwidth}{0.5\textwidth}
\setlength{\glspagelistwidth}{0.15\textwidth}

\makeglossaries

\renewcommand*{\glsgroupskip}{}

\loadglsentries{NotesonProjectsPartI-Znotation}


%\renewcommand{\jot}{2\jot}

% For command \textdegree, \textcelsius, ect.

\DeclareMathOperator \J {J}

\ifpdf
\graphicspath{{./}{./Fig/pdf/}{./Fig/}}
\else
\graphicspath{{./}{./Fig/eps/}{./Fig/}}
\fi

\makeatother

\usepackage{babel}

\begin{document}

\title{Notes on Inverted Pendulum System Laboratory}


\author{Dazhi Jiang}

\maketitle

\runtitle{Notes on Inverted Pendulum System Laboratory}


\runauthor{Dazhi Jiang}


\section{Introduction}

This document includes some notes on the inverted pendulum system
laboratory (ACS220).


\section{Notes}


\subsection{The model of inverted pendulum system}


\subsubsection{Schematic diagram}

Figure~\ref{fig:InvertedPendulum} shows a schematic drawing of the
inverted pendulum system.



%
\begin{figure}[!ht]
\begin{centering}
%\tikzset{external/export=true}
%\tikzsetnextfilename{InvertedPendulum}
%% -*- Mode: LaTeX Memoir; tab-width: 4;

%% InvertedPendulum.tikz    Inverted Pendulum
%% Draw an inverted pendulum using PGF/TikZ
%% MAIN FILE: TeX Main File Name
%% Created by Dazhi Jiang, 2011-02-01 16:49:22 +0000 (Tue,  1 Feb 2011)
%% Copyright (c) 2011 Dazhi Jiang. All Rights Reserved. 

%% TextMate Settings
%!TEX root = TeX Main File Name

\tikzset{%
	interface/.style={
		% The border decoration is a path replacing decorator. 
		% For the interface style we want to draw the original path.
		% The postaction option is therefore used to ensure that the
		% border decoration is drawn *after* the original path.
		postaction={draw, decorate, decoration={border, angle=-45,
					amplitude=0.3cm, segment length=2mm}}},
	helparrow/.style={>=latex', blue, thick},
	% helparrow/.style={>=latex', draw=blue, fill=blue, very thick},
	helpline/.style={thin, black!90, opacity=0.5},
	force/.style={>=stealth, draw=blue, fill=blue, ultra thick},
}

\def\ground{%
	\fill [black!20] (0, 0) rectangle (49mm, -13mm);
	\fill [black!20] (49mm, 0) rectangle (80mm, -5mm);
	\draw [thick, black!80, interface] (0, 0) -- (80mm, 0);
	\draw [thick, black!80] (49mm, 0) -- +(0, -13mm);
	\draw [|->, helparrow] (49mm, -9mm) -- ++(20mm, 0) node [right] {$x$};
}

\def\cart{
	\filldraw [%thick,%
		draw = black!80,%
		fill = black!40,%
		top color = black!10,%
		bottom color = black!40,%
		% pattern=horizontal lines gray,%
		] (0,0) rectangle (30mm, 10mm);
	\node at (15mm, 5mm) {$M$};
	\draw[->, force] (-15mm, 5mm) node [above] {$\vv{F}$} -- (0, 5mm);
}

\def\pendulum{%
	\filldraw [%thick,%
		draw = black!80,%
		% fill = black!25,%
		left color=black!10,%
		bottom color=black!40,%
		% pattern=horizontal lines gray,%
		] (-0.8mm, 0) rectangle (0.8mm, 20mm);
}

\def\joint{%
	\filldraw [%thick,%
		draw = black!80,%
		fill = white,%
		% pattern=horizontal lines gray,%
		] (0, 0) circle (1mm);
}

\def\wheel{%
	\fill [thin,%
		fill = black!70,%
		path fading = south] (0, 0) circle (1.75mm);
	\begin{scope}
		\clip (0, 0) circle (1.75mm);
		\fill [fill=black!30,] (0, -1mm) circle (2mm);
	\end{scope}
	\fill [fill=black!90] (0, 0) circle (0.5mm);
	\draw [thin,%
		double = black!20,%
		double distance = 0.5mm] (0, 0) circle (2mm);
}


%\tikzsetnextfilename{PIDBode}
\begin{tikzpicture}[> = latex',%
					scale = 1,%
					text = blue]
	\begin{scope}
		\wheel
	\end{scope}
	\begin{scope}[xshift=18mm]
		\wheel
	\end{scope}
	\begin{scope}[xshift=-31mm, yshift=-2.4mm]
		\ground
	\end{scope}
	\begin{scope}[shift = {(-6mm, 2.4mm)}]
		\cart
	\end{scope}
	\begin{scope}[shift = {(9mm, 12.4mm)}]
		\draw [helpline] (0, -20mm) -- (0, 25mm);
		\draw [helpline] (110:20.5mm) -- (110:25mm);
		\draw [|->, helparrow] (0, 23mm) 
			arc [radius=23mm, start angle=90, delta angle=20] ;
		\node at (97:25mm) {$\theta'$};
		\draw [|->, helparrow] (0, -9mm) 
			arc [radius=9mm, start angle=-90, delta angle=195] ;
		\node [above right] at (50:8.5mm) {$\theta$};
		\draw [thin] (110:14mm) -- (-10mm, 10mm) 
			node [left] {$m$, $I$};
	\end{scope}
	\begin{scope}[shift = {(9mm, 12.4mm)}, rotate=20]
		\pendulum
	\end{scope}
	\begin{scope}[shift = {(9mm, 12.4mm)}]
		\joint
	\end{scope}
\end{tikzpicture}
\par\end{centering}

\centering{}\caption{A schematic diagram of the inverted pendulum system. }
\label{fig:InvertedPendulum}%
\end{figure}



\subsubsection{Equations of motion}

The motion of the cart in the horizontal direction ($x$-direction)
is described by

\begin{equation}
M\ddot{x}+b\dot{x}+\vv{N}=\vv{F},\label{eq:CartMotion}\end{equation}
where $\vv{N}$ is an unknown reaction force applied by the pendulum
in the $x$-direction, which is given by\begin{equation}
\vv{N}=m\ddot{x}+ml\ddot{\theta}\cos\theta-ml\dot{\theta}^{2}\sin\theta.\label{eq:ReactionForcex}\end{equation}


The motion of the pendulum is described by\begin{equation}
(I+ml^{2})\ddot{\theta}+mgl\sin\theta=-ml\ddot{x}\cos\theta.\label{eq:PendulumMotion}\end{equation}


By combining Equations~\ref{eq:CartMotion} and \ref{eq:ReactionForcex},
$\vv{N}$ can be eliminated to yield\begin{equation}
(M+m)\ddot{x}+b\dot{x}+ml\ddot{\theta}\cos\theta-ml\dot{\theta}^{2}\sin\theta=\vv{F}.\label{eq:CartMotionNoN}\end{equation}



\subsubsection{Linearisation}

Equations~\ref{eq:PendulumMotion} and \ref{eq:CartMotionNoN} are
the nonlinear differential equations and can be linearised around
their equilibrium points. Assume $\theta=\pi+\theta'$ around the
unstable equilibrium of the system (\ie $\theta\approx\pi$ or $\theta'\approx0$),
where $\theta'$ represents motion from the vertical upward direction.
Let $\cos\theta\approx-1$, $\sin\theta\approx-\theta'$, and $\dot{\theta}^{2}\approx0$.
Thus, Equations~\ref{eq:PendulumMotion} and \ref{eq:CartMotionNoN}
are approximated by

\begin{eqnarray}
(I+ml^{2})\ddot{\theta'}-mgl\theta' & = & ml\ddot{x},\label{eq:LinearPM}\\
(M+m)\ddot{x}+b\dot{x}-ml\ddot{\theta'} & = & \vv{F}.\label{eq:LinearCM}\end{eqnarray}



\subsubsection{Laplace transform}

By applying the Laplace transform to both sides of the Equations~\ref{eq:LinearPM}
and \ref{eq:LinearCM} , we obtain\begin{eqnarray}
(I+ml^{2})s^{2}\Theta(s)-mgl\Theta(s) & = & mls^{2}X(s),\\
{}[(M+m)s^{2}+bs]X(s)-mls^{2}\Theta(s) & = & F(s),\end{eqnarray}
where assume all the initial conditions to be zero. Thus the transfer
function from the control input, $\vv{F}$, to the output, $\theta'$
is given by\begin{eqnarray}
\frac{\Theta'(s)}{F(s)} & = & \frac{mls^{2}}{[(M+m)(I+ml^{2})-m^{2}l^{2}]s^{4}+b(I+ml^{2})s^{3}-mgl(M+m)s^{2}-bmgls}\\
 & = & \frac{mls}{[(M+m)(I+ml^{2})-m^{2}l^{2}]s^{3}+b(I+ml^{2})s^{2}-mgl(M+m)s-bmgl}\\
 & = & \frac{\dfrac{ml}{C}s}{s^{3}+\dfrac{b(I+ml^{2})}{C}s^{2}-\dfrac{mgl(M+m)}{C}s-\dfrac{bmgl}{C}},\end{eqnarray}
 where \begin{equation}
C=(M+m)(I+ml^{2})-m^{2}l^{2}.\end{equation}



\subsubsection{Stability analysis}

Substituting the value of given variables, the transfer function becomes\begin{equation}
G(s)=\frac{\Theta'(s)}{F(s)}=\frac{0.8045s}{s^{3}+1.726s^{2}-10.58s-18}.\end{equation}
Covert it to \emph{zeor-pole-gain} form,\begin{equation}
G(s)=\frac{\Theta'(s)}{F(s)}=\frac{0.8045s}{(s-3.244)(s+3.277)(s+1.694)}.\end{equation}
Clearly, the system is unstable.


\subsubsection{Lead and Lag compensator}

Assume the transfer function of lead (lag) compensator is\[
C(s)=\frac{s+\alpha}{s+\beta},\]
and for later convenience the system transfer function is\begin{equation}
G(s)=\frac{\Theta'(s)}{F(s)}=\frac{ks}{s^{3}+a_{1}s+a_{2}s-a_{3}}.\end{equation}
Thus the close-loop transfer function can be calculated by\begin{equation}
G_{c}=\frac{ks(s+\alpha)}{s^{4}+(a_{1}+\beta)s^{3}+(a_{1}\beta+a_{2}+k)s^{2}+(a_{2}\beta+a_{3}+\alpha k)s+a_{3}\beta}.\end{equation}



\paragraph{Routh\textendash{}Hurwitz stability criterion}

%
\begin{table}[!tbh]
\caption{Routh\textendash{}Hurwitz stability criterion for Lead (Lag) compensator}


\begin{centering}
\begin{tabular}{lll}
\toprule 
1 & $a_{1}\beta+a_{2}+k$ & $a_{3}\beta$\tabularnewline
\addlinespace
$a_{1}+\beta$ & $a_{2}\beta+a_{3}+\alpha k$ & $0$\tabularnewline
\addlinespace
$q_{1}$ & $a_{3}\beta$ & \tabularnewline
\addlinespace
$q_{2}$ & 0 & \tabularnewline
\addlinespace
$a_{3}\beta$ &  & \tabularnewline
\bottomrule
\addlinespace
\end{tabular}
\par\end{centering}

%
\end{table}
Since $a_{3}<0$ and $\beta>0$, $a_{3}\beta<0$. This implies that
there is at least one pole (positive) located in the right half-plane.


\subsubsection{Lead-lag compensator}

Similarly, it is impossible to make the system stable using the lead-lag
compensator. This is because the last term of the dominator of the
transfer function of the close-loop system is negtive (this implies
that there is at least one positive pole).


\subsubsection{PID}

PID controller can do the job


\subsection{Cart Motion Control}


\subsubsection{Identification}
\begin{enumerate}
\item Open \emph{Pendulum\_Identification.mdl.}
\item Click \emph{Tools$\rightarrow$Control Design$\rightarrow$Linear
Analysis}.
\item In the tab \emph{Linearization}, click the button \emph{Linearize
Model}.
\item Click \emph{Model} in the left panel to check the linearization result.
\item Click the button \emph{Export to Workspace} to export the model to
workspace.
\end{enumerate}

\subsubsection{PID controller design}
\begin{enumerate}
\item Open \emph{CartControl.mdl}.
\item Build the model
\item Adjust the parameters.
\end{enumerate}

\section{Some Keywords}
\begin{itemize}
\item inverted pendulum
\end{itemize}

\section{Conclusions}

{
\renewcommand{\arraystretch}{1.1} % Vertical space between table lines
\renewcommand*{\symbolname}{Value}
\printglossary[title=List of Notations, toctitle=List of Notations]
}


\section*{Tmp}

A bug of asmmath, can be fixed by mathtools

\[
\begin{gathered}{}[p]=100\\
{}[v]=200\end{gathered}
\]

\end{document}
