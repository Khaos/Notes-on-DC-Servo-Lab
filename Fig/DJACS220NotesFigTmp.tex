%% LyX 1.6.5 created this file.  For more info, see http://www.lyx.org/.
%% Do not edit unless you really know what you are doing.
\documentclass[10pt,a4paper,british]{article}
\usepackage[]{fontenc}
\usepackage[latin9]{inputenc}
\usepackage{amstext}

\makeatletter
%%%%%%%%%%%%%%%%%%%%%%%%%%%%%% Textclass specific LaTeX commands.
 \usepackage[headings]{DJespcrc1}

%%%%%%%%%%%%%%%%%%%%%%%%%%%%%% User specified LaTeX commands.
%\input{DJcommonpk-phd1-pgf-en}
%\input{DJcommonpk-pgf-en}

% Definition of command \djphd
\input{DJ4DJespcrc}

% Commonly used mathematical macro
%\input{DJcommonmath}
% identification
%\readRCS $Id: ModifiedFICA.tex,v 0.01 2009/06/15 15:20:11 Exp $
%\ProvidesFile{ModifiedFICA.tex}[\filedate \space v\fileversion
%     \space DJElsevier 1-column CRC Author Instructions]
\readRCS $Id: DJACS220Notes.tex,v 0.01 2010/02/20 15:20:11 Exp $
\ProvidesFile{DJACS220Notes.tex}[\filedate \space v\fileversion
     \space DJElsevier 1-column CRC Author Instructions]



% This package is not declared in DJespcrc1.sty
%\usepackage[numbers, sort&compress]{natbib}
% pdfsync.sty allows one to synchronize between LaTeX source and pdf output
%\usepackage{pdfsync}




% change this to the following line for use with LaTeX2.09
% \documentstyle[12pt,twoside,fleqn,espcrc1]{article}
% if you want to include PostScript figures
%\usepackage{graphicx}
% if you have landscape tables
%\usepackage[figuresright]{rotating}
% Put all the style files and packages you want here
% Load DJ-defined report packages
\input{DJReportLoadAll}

%\usepackage{amsmath}

\usepackage{mathtools}

\usepackage{djschemabloc}

\usepackage{bodegraph}

\usetikzlibrary{circuits}

\usetikzlibrary{intersections} 

%\usetikzlibrary{positioning}

%\usetikzlibrary{external}
%\tikzexternalize{DJACS220NotesFigTmp} % provide the file's real name


%\renewcommand{\jot}{2\jot}

% For command \textdegree, \textcelsius, ect.

\DeclareMathOperator \J {J}

\ifpdf
\graphicspath{{./}{./Fig/pdf/}{./Fig/}}
\else
\graphicspath{{./}{./Fig/eps/}{./Fig/}}
\fi

\makeatother

\usepackage{babel}

\begin{document}

\title{Notes on DC Servo Lab}


\author{Dazhi Jiang}

\maketitle

\runtitle{Notes on DC Servo Lab}


\runauthor{Dazhi Jiang}


\section{Notes}

%
% Define the layers for the diagram
\pgfdeclarelayer{background}
\pgfdeclarelayer{foreground}
\pgfsetlayers{background,main,foreground}

%\tikzsetnextfilename{PIDBode}
\begin{tikzpicture}[>=latex', 
    ref lines/.style={thin, blue!60}, 
    ref points/.style={circle, black, opacity=0.7, fill, minimum size= 3pt, inner sep=0}, 
    every node/.style={font=\small}, 
    bode lines/.style={very thick, blue}, 
    Gclabel/.style={text=blue}, 
    xscale=12/3]
    
\begin{scope}[yscale=4/110]
\UnitedB
\semilog{-1}{2}{-50}{60}

% Bode plot (magnitude) for the original system, 4/(s/(1+2s)). 
% Asymptotic line
\BodeAmp[ref lines, red!60]{-1:1.35}{\POAmpAsymp{4}{2.0}+\IntAmp{1}}
% Bode plot
\BodeAmp[bode lines, red, name path=Gomagnitude]{-1:1.35}{\POAmp{4}{2.0}+\IntAmp{1}}

% Bode plot (magnitude) for the original system, 4(1+s)/(s^2/(1+0.1s)). 
% Asymptotic line
\BodeAmp[ref lines]{-1:2}{\APAmpAsymp{4}{0.1}{10}+2*\IntAmp{1}}
% Bode plot
\BodeAmp[bode lines, name path=magnitude]{-1:2}{\APAmp{4}{0.1}{10}+2*\IntAmp{1}}

% Reference line, (0dB)
\draw [name path=unitygain, ref lines] (-1,0) -- (2,0);

% Crossover frequency of the original system
\path [name intersections={of=magnitude and unitygain, by=crossover}];
\node (coref) [ref points, label={[blue]below left:$\omega_c$}] at (crossover) {};

% Crossover frequency of the compensated system
\path [name intersections={of=Gomagnitude and unitygain, by=Gocrossover}];
\node (Gocoref) [ref points, label={[red]below:$\omega_{co}$}] at (Gocrossover) {};

% Labels for the original system (open-loop)
\node [Gclabel, text=red] at (-0.7, 15) {$-20$dB/dec};
\node [Gclabel, text=red] at (0.5, -30) {$-40$dB/dec};
\node (KG) [Gclabel, red!60, text=red, draw] at (-0.5, -30) {$KG(s)=\dfrac{4}{s(1+2s)}$};
\draw (KG.east) edge [->, shorten >=1pt, thick, red, bend right=1.5] (0.4, -10);

% Labels for the compensated system (open-loop)
\node [ref points, label={[blue]below left:$\omega_2$}] at (0, 0) {};
\node [ref points, label={[blue]below right:$\omega_3$}] at (1, 0) {};
\node [Gclabel] at (-0.4, 40) {$-40$dB/dec};
\node [Gclabel] at (0.5, 10) {$-20$dB/dec};
\node [Gclabel] at (1.6, -20) {$-40$dB/dec};
\node (KDG) [Gclabel, blue!60, text=blue, draw] at (1.4, 40) {$KD(s)G(s)=\dfrac{4(1+s)}{s^2(1+0.1s)}$};
\draw (KDG.west) edge [->, shorten >=1pt, thick, blue, bend right=1.5] (0.17, 10);

% Axes label
\node [below=6pt] at (0.5,-50) {Frequency, $\omega$};
\node [rotate=90] at (-1.25,5) {Magnitude, $20\log(|G(\text{j}\omega)|)$};


\end{scope}

\begin{scope}[yshift=-3.5cm,yscale=4/180]
\UniteDegre
\OrdBode{30}
\semilog{-1}{2}{-180}{0}
% Bode plot (phase) for the original system, 4/(s/(1+2s)). 
% Asymptotic line
\BodeArg[ref lines, red!60]{-1:2}{\POArgAsymp{4}{2.0}+\IntArg{1}}
% Bode plot
\BodeArg[bode lines, red, name path=Gophase]{-1:2}{\POArg{4}{2}+\IntArg{1}}

% Bode plot (magnitude) for the original system, 4(1+s)/(s^2/(1+0.1s)). 
% Asymptotic line
\BodeArg[ref lines]{-1:2}{\APArgAsymp{4}{0.1}{10}+2*\IntArg{1}}
% Bode plot
\BodeArg[bode lines, name path=phase]{-1:2}{\APArg{4}{0.1}{10}+2*\IntArg{1}}

% Phase margin of the original system
\path [name path=Gowcref] (Gocrossover) -- ++(0,-330);
\path [name intersections={of=Gophase and Gowcref, by=Gophasemargin}];
\node (Gopmref) [ref points] at (Gophasemargin) {};
\draw [ref lines, red!60, densely dotted] (Gocoref.south) -- (Gopmref.north);
\draw [ref lines, <->, red] let \p1=(Gophasemargin)
    in
        (Gopmref.south) -- node[left, Gclabel, text=red] {$\text{PM}_o$} (\x1,-180);
        
% Phase margin of the compensated system
\path[name path=wcref] (crossover) -- ++(0,-300);
\path [name intersections={of=phase and wcref, by=phasemargin}];
\node (pmref) [ref points] at (phasemargin) {};
\draw [ref lines, densely dotted] (coref.south) -- (pmref.north);
\draw [ref lines, <->, blue] let \p1=(phasemargin)
    in
        (pmref.south) -- node[left, Gclabel] {PM} (\x1,-180);

       
\node (KGphase) [Gclabel, red!60, text=red, draw] at (-0.5, -40) {$KG(s)=\dfrac{4}{s(1+2s)}$};
\draw[->, shorten >=1pt, thick, red] (KGphase.south) .. controls +(down:30) and +(0.1,10) .. (-0.65, -114);

\node (KDGphase) [Gclabel, blue!60, text=blue, draw] at (1.4, -40) {$KD(s)G(s)=\dfrac{4(1+s)}{s^2(1+0.1s)}$};
\draw[->, shorten >=1pt, thick, blue] (KDGphase.south) .. controls +(down:40) and +(0.1,30) .. (1.1, -146);

% Axes label        
\node [below=6pt] at (0.5, -180) {Frequency, $\omega$};
\node [rotate=90] at (-1.25, -90) {Phase, $\angle G(\text{j}\omega)$};
\end{scope}

% Background, try tkzexample later.
\begin{pgfonlayer}{background}
    \path (-1.4cm,2.8cm) node (tl) {};
    \path (2.3cm, -8.4cm) node (br) {};
    
    \path[fill=brown!20] (tl) rectangle (br);
\end{pgfonlayer}

\end{tikzpicture}


\section{Conclusions}


\section*{Tmp}

A bug of asmmath, can be fixed by mathtools

\[
\begin{gathered}{}[p]=100\\
{}[v]=200\end{gathered}
\]

\end{document}
