%% LyX 1.6.5 created this file.  For more info, see http://www.lyx.org/.
%% Do not edit unless you really know what you are doing.
\documentclass[10pt,a4paper,british]{article}
\usepackage[]{fontenc}
\usepackage[latin9]{inputenc}
\usepackage{amstext}

\makeatletter
%%%%%%%%%%%%%%%%%%%%%%%%%%%%%% Textclass specific LaTeX commands.
 \usepackage[headings]{DJespcrc1}

%%%%%%%%%%%%%%%%%%%%%%%%%%%%%% User specified LaTeX commands.
%\input{DJcommonpk-phd1-pgf-en}
%\input{DJcommonpk-pgf-en}

% Definition of command \djphd
\input{DJ4DJespcrc}

% Commonly used mathematical macro
%\input{DJcommonmath}
% identification
%\readRCS $Id: ModifiedFICA.tex,v 0.01 2009/06/15 15:20:11 Exp $
%\ProvidesFile{ModifiedFICA.tex}[\filedate \space v\fileversion
%     \space DJElsevier 1-column CRC Author Instructions]
\readRCS $Id: DJACS220Notes.tex,v 0.01 2010/02/20 15:20:11 Exp $
\ProvidesFile{DJACS220Notes.tex}[\filedate \space v\fileversion
     \space DJElsevier 1-column CRC Author Instructions]



% This package is not declared in DJespcrc1.sty
%\usepackage[numbers, sort&compress]{natbib}
% pdfsync.sty allows one to synchronize between LaTeX source and pdf output
%\usepackage{pdfsync}




% change this to the following line for use with LaTeX2.09
% \documentstyle[12pt,twoside,fleqn,espcrc1]{article}
% if you want to include PostScript figures
%\usepackage{graphicx}
% if you have landscape tables
%\usepackage[figuresright]{rotating}
% Put all the style files and packages you want here
% Load DJ-defined report packages
\input{DJReportLoadAll}

%\usepackage{amsmath}

\usepackage{mathtools}
\usepackage{esvect}

\usetikzlibrary{circuits}

\usetikzlibrary{intersections} 

\usetikzlibrary{fadings, patterns}

\usetikzlibrary{%
    decorations.pathreplacing,%
    decorations.pathmorphing%
}

\usetikzlibrary{positioning}

%\usetikzlibrary{positioning}

%\usetikzlibrary{external}
%\tikzexternalize{DJACS220NotesFigTmp} % provide the file's real name


%\renewcommand{\jot}{2\jot}

% For command \textdegree, \textcelsius, ect.

\DeclareMathOperator \J {J}

\ifpdf
\graphicspath{{./}{./Fig/pdf/}{./Fig/}}
\else
\graphicspath{{./}{./Fig/eps/}{./Fig/}}
\fi

\makeatother

\usepackage{babel}

\begin{document}

\title{Notes on Inverted Pendulum}


\author{Dazhi Jiang}

\maketitle

\runtitle{Notes on DC Servo Lab}


\runauthor{Dazhi Jiang}


\section{Notes}

\tikzset{%
	interface/.style={
		% The border decoration is a path replacing decorator. 
		% For the interface style we want to draw the original path.
		% The postaction option is therefore used to ensure that the
		% border decoration is drawn *after* the original path.
		postaction={draw, decorate, decoration={border, angle=-45,
					amplitude=0.3cm, segment length=2mm}}},
	helparrow/.style={>=latex', draw=blue, fill=blue, very thick},
	force/.style={>=latex', draw=blue, fill=blue, ultra thick},
}

\def\ground{%
	\fill [black!20] (0, 0) rectangle (49mm, -13mm);
	\fill [black!20] (49mm, 0) rectangle (80mm, -5mm);
	\draw [thick, black!80, interface] (0, 0) -- (80mm, 0);
	\draw [thick, black!80] (49mm, 0) -- +(0, -13mm);
	\draw [|->, helparrow] (49mm, -9mm) -- ++(20mm, 0) node [right] {$x$};
}

\def\cart{
	\filldraw [%thick,%
		draw = black!80,%
		fill = black!40,%
		% pattern=horizontal lines gray,%
		] (0,0) rectangle (30mm, 10mm);
	\node at (15mm, 5mm) {$M$};
	\draw[->, force] (-15mm, 5mm) node [above] {$\vv{F}$} -- (0, 5mm);
}

\def\pendulum{%
	\filldraw [%thick,%
		draw = black!80,%
		fill = black!25,%
		% pattern=horizontal lines gray,%
		] (-0.8mm, 0) rectangle (0.8mm, 20mm);
}

\def\joint{%
	\filldraw [%thick,%
		draw = black!80,%
		fill = white,%
		% pattern=horizontal lines gray,%
		] (0, 0) circle (1mm);
}

\def\wheel{%
	\fill [thin,%
		fill = black!70,%
		path fading = south] (0, 0) circle (1.75mm);
	\begin{scope}
		\clip (0, 0) circle (1.75mm);
		\fill [fill=black!30,] (0, -1mm) circle (2mm);
	\end{scope}
	\fill [fill=black!90] (0, 0) circle (0.5mm);
	\draw [thin,%
		double = black!20,%
		double distance = 0.5mm] (0, 0) circle (2mm);
}


%\tikzsetnextfilename{PIDBode}
\begin{tikzpicture}[> = latex', %
					scale = 1.5]
	\begin{scope}
		\wheel
	\end{scope}
	\begin{scope}[xshift=18mm]
		\wheel
	\end{scope}
	\begin{scope}[xshift=-31mm, yshift=-2.4mm]
		\ground
	\end{scope}
	\begin{scope}[shift = {(-6mm, 2.4mm)}]
		\cart
	\end{scope}
	\begin{scope}[shift = {(9mm, 12.4mm)}]
		\draw [thin, blue!70, opacity=0.5] (0, -20mm) -- (0, 20mm);
		\draw [->, helparrow, shorten >=0.8mm] (0, 18mm) 
			arc [radius=18mm, start angle=90, delta angle=20] ;
		\node at (97:21mm) {$\theta'$};
		\node [above left] at (120:15mm) {$m$, $I$};
		% \draw [->, helparrow, shorten >=0.8mm] (0, 18mm) 
		% 	arc [radius=18mm, start angle=90, delta angle=20] ;
		% \node at (97:22mm) {$\theta'$};
	\end{scope}
	\begin{scope}[shift = {(9mm, 12.4mm)}, rotate=20]
		\pendulum
	\end{scope}
	\begin{scope}[shift = {(9mm, 12.4mm)}]
		\joint
	\end{scope}
\end{tikzpicture}


\end{document}
