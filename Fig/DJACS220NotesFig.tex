%% LyX 1.6.5 created this file.  For more info, see http://www.lyx.org/.
%% Do not edit unless you really know what you are doing.
\documentclass[12pt,a4paper,british]{article}
\usepackage[]{fontenc}
\usepackage[latin9]{inputenc}
\usepackage{amstext}

\makeatletter
%%%%%%%%%%%%%%%%%%%%%%%%%%%%%% Textclass specific LaTeX commands.
 \usepackage[headings]{DJespcrc1}

%%%%%%%%%%%%%%%%%%%%%%%%%%%%%% User specified LaTeX commands.
%\input{DJcommonpk-phd1-pgf-en}
%\input{DJcommonpk-pgf-en}

% Definition of command \djphd
\input{DJ4DJespcrc}

% Commonly used mathematical macro
%\input{DJcommonmath}
% identification
%\readRCS $Id: ModifiedFICA.tex,v 0.01 2009/06/15 15:20:11 Exp $
%\ProvidesFile{ModifiedFICA.tex}[\filedate \space v\fileversion
%     \space DJElsevier 1-column CRC Author Instructions]
\readRCS $Id: DJACS220Notes.tex,v 0.01 2010/02/20 15:20:11 Exp $
\ProvidesFile{DJACS220Notes.tex}[\filedate \space v\fileversion
     \space DJElsevier 1-column CRC Author Instructions]



% This package is not declared in DJespcrc1.sty
%\usepackage[numbers, sort&compress]{natbib}
% pdfsync.sty allows one to synchronize between LaTeX source and pdf output
%\usepackage{pdfsync}




% change this to the following line for use with LaTeX2.09
% \documentstyle[12pt,twoside,fleqn,espcrc1]{article}
% if you want to include PostScript figures
%\usepackage{graphicx}
% if you have landscape tables
%\usepackage[figuresright]{rotating}
% Put all the style files and packages you want here
% Load DJ-defined report packages
\input{DJReportLoadAll}

%\usepackage{amsmath}

\usepackage{mathtools}

\usepackage{djschemabloc}


\usetikzlibrary{circuits}

\usetikzlibrary{external}
\tikzexternalize{DJACS220NotesFig} % provide the file's real name


%\renewcommand{\jot}{2\jot}

% For command \textdegree, \textcelsius, ect.

\DeclareMathOperator \J {J}

\ifpdf
\graphicspath{{./}{./Fig/pdf/}{./Fig/}}
\else
\graphicspath{{./}{./Fig/eps/}{./Fig/}}
\fi

\makeatother

\usepackage{babel}

\begin{document}

\title{Notes on DC Servo Lab}


\author{Dazhi Jiang}

\maketitle

\runtitle{Notes on DC Servo Lab}


\runauthor{Dazhi Jiang}


\section{Introduction}


\section{Vocabulary}


\section{Notes}


\subsection{Armature controlled motor model}

Figure~\ref{fig:MotorModel} shows the simplified block diagram of
control system for armature controlled motor

%
\tikzsetnextfilename{Motor_Model}
\begin{tikzpicture}[every node/.style={font=\small}]
    \sbStyleBloc{minimum width=3em, fill=yellow!20, thick}
    \sbStyleSum{minimum size=1em, fill=black!50, thick}
    \sbEntree{input}
    \sbComp*[4]{comp}{input}
        \sbRelier{input}{comp}
        \sbNomLien[0.8]{input}{$V_a$}
    \sbBloc[3]{system}{$\dfrac{K}{1+sT}$}{comp}
        \sbRelier[$e$]{comp}{system}
    \sbSortie[4]{output}{system}
        \sbRelier{system}{output}
        \sbNomLien[0.8]{output}{$\omega$}
    \sbRenvoi{system-output}{comp}{}
    
\end{tikzpicture}


\subsection{Position feedback control}

Figure~\ref{fig:PositionFeedback} shows the block diagram for the
position feedback servo system.

%
\tikzsetnextfilename{Position_Feedback}
\begin{tikzpicture}[every node/.style={font=\small}]
    \sbStyleSum{minimum size=1em, fill=black!50, thick}
    \sbEntree{input}
    \sbComp*[4]{comp}{input}
        \sbRelier{input}{comp}
        \sbNomLien[0.8]{input}{$\theta_i$}
    \sbStyleBloc{minimum width=3em, fill=red!20, thick}
    \sbBloc[3]{gain}{$K_a$}{comp}
        \sbRelier[$e$]{comp}{gain}
    \sbStyleBloc{minimum width=3em, fill=yellow!20, thick}
    \sbBloc[3]{system}{$\dfrac{K}{s(1+sT)}$}{gain}
        \sbRelier{gain}{system}
    \sbSortie[4]{output}{system}
        \sbRelier{system}{output}
        \sbNomLien[0.8]{output}{$\theta_o$}
    \sbRenvoi{system-output}{comp}{}
    
\end{tikzpicture}


\subsection{Velocity feedback control}

Figure~\ref{fig:VelocityFeedback} shows the block diagram for the
position feedback servo system.

%
\tikzsetnextfilename{Velocity_Feedback}
\begin{tikzpicture}[every node/.style={font=\small}]
    \sbStyleSum{minimum size=1em, fill=black!50, thick}
    \sbEntree{input}
    \sbCompSum*[4]{comp}{input}{-}{-}{+}{}
        \sbRelier{input}{comp}
        \sbNomLien[0.8]{input}{$\theta_i$}
    \sbStyleBloc{minimum width=3em, fill=red!20, thick}
    \sbBloc[3]{gain}{$K_a$}{comp}
        \sbRelier[$e$]{comp}{gain}
    \sbStyleBloc{minimum width=3em, fill=yellow!20, thick}
    \sbBloc[3]{system}{$\dfrac{K}{(1+sT)}$}{gain}
        \sbRelier{gain}{system}
    \sbBloc[3]{integral}{$\dfrac{1}{s}$}{system}
        \sbRelier{system}{integral}
    \sbSortie[4]{output}{integral}
        \sbRelier{integral}{output}
        \sbNomLien[0.8]{output}{$\theta_o$}
    \sbRenvoi{integral-output}{comp}{}
    \sbDecaleNoeudy[-4]{system}{VfRef}
    \sbStyleBloc{minimum width=3em, fill=red!20, thick}
    \sbBlocr[3]{Vf}{$K_f$}{VfRef}
    \sbRelieryx{system-integral}{Vf}
    \sbRelierxy{Vf}{comp}
    
\end{tikzpicture}


\subsection{Velocity control system}

Figure~\ref{fig:VelocityControl} shows the block diagram for the
speed controlled system.

%
\tikzsetnextfilename{Velocity_Control}
\begin{tikzpicture}[every node/.style={font=\small}]
    \sbStyleSum{minimum size=1em, fill=black!50, thick}
    \sbEntree{input}
    \sbCompSum*[4]{comp}{input}{-}{}{+}{}
        \sbRelier{input}{comp}
        \sbNomLien[0.8]{input}{$\omega_i$}
    \sbStyleBloc{minimum width=3em, fill=red!20, thick}
    \sbBloc[3]{gain}{$K_a$}{comp}
        \sbRelier[$e$]{comp}{gain}
    \sbStyleBloc{minimum width=3em, fill=yellow!20, thick}
    \sbBloc[3]{system}{$\dfrac{K}{(1+sT)}$}{gain}
        \sbRelier{gain}{system}
    \sbSortie[4]{output}{system}
        \sbRelier{system}{output}
        \sbNomLien[0.8]{output}{$\omega_o$}
    \sbDecaleNoeudy[-4]{system}{VfRef}
    \sbStyleBloc{minimum width=3em, fill=red!20, thick}
    \sbBlocr[3]{Vf}{$K_f$}{VfRef}
    \sbRelieryx{system-output}{Vf}
    \sbRelierxy{Vf}{comp}
    
\end{tikzpicture}

\section{Conclusions}


\section*{Tmp}

A bug of asmmath, can be fixed by mathtools

\[
\begin{gathered}{}[p]=100\\
{}[v]=200\end{gathered}
\]

\end{document}
