%% LyX 1.6.5 created this file.  For more info, see http://www.lyx.org/.
%% Do not edit unless you really know what you are doing.
\documentclass[10pt,a4paper,british]{article}
\usepackage[]{fontenc}
\usepackage[latin9]{inputenc}
\usepackage{amstext}

\makeatletter
%%%%%%%%%%%%%%%%%%%%%%%%%%%%%% Textclass specific LaTeX commands.
 \usepackage[headings]{DJespcrc1}

%%%%%%%%%%%%%%%%%%%%%%%%%%%%%% User specified LaTeX commands.
%\input{DJcommonpk-phd1-pgf-en}
%\input{DJcommonpk-pgf-en}

% Definition of command \djphd
\input{DJ4DJespcrc}

% Commonly used mathematical macro
%\input{DJcommonmath}
% identification
%\readRCS $Id: ModifiedFICA.tex,v 0.01 2009/06/15 15:20:11 Exp $
%\ProvidesFile{ModifiedFICA.tex}[\filedate \space v\fileversion
%     \space DJElsevier 1-column CRC Author Instructions]
\readRCS $Id: DJACS220Notes.tex,v 0.01 2010/02/20 15:20:11 Exp $
\ProvidesFile{DJACS220Notes.tex}[\filedate \space v\fileversion
     \space DJElsevier 1-column CRC Author Instructions]



% This package is not declared in DJespcrc1.sty
%\usepackage[numbers, sort&compress]{natbib}
% pdfsync.sty allows one to synchronize between LaTeX source and pdf output
%\usepackage{pdfsync}




% change this to the following line for use with LaTeX2.09
% \documentstyle[12pt,twoside,fleqn,espcrc1]{article}
% if you want to include PostScript figures
%\usepackage{graphicx}
% if you have landscape tables
%\usepackage[figuresright]{rotating}
% Put all the style files and packages you want here
% Load DJ-defined report packages
\input{DJReportLoadAll}

%\usepackage{amsmath}

\usepackage{mathtools}

\usepackage{djschemabloc}

\usepackage{bodegraph}

\usetikzlibrary{circuits}

\usetikzlibrary{intersections} 

%\usetikzlibrary{positioning}

\usetikzlibrary{external}
\tikzexternalize{DJACS220NotesFig} % provide the file's real name


%\renewcommand{\jot}{2\jot}

% For command \textdegree, \textcelsius, ect.

\DeclareMathOperator \J {J}

\ifpdf
\graphicspath{{./}{./Fig/pdf/}{./Fig/}}
\else
\graphicspath{{./}{./Fig/eps/}{./Fig/}}
\fi

\makeatother

\usepackage{babel}

\begin{document}

\title{Notes on DC Servo Lab}


\author{Dazhi Jiang}

\maketitle

\runtitle{Notes on DC Servo Lab}


\runauthor{Dazhi Jiang}


\section{Introduction}


\section{Vocabulary}


\section{Notes}


\subsection{Armature controlled motor model}

Figure~\ref{fig:MotorModel} shows the simplified block diagram of
control system for armature controlled motor

%
\tikzsetnextfilename{Motor_Model}
\begin{tikzpicture}[every node/.style={font=\small}]
    \sbStyleBloc{minimum width=3em, fill=yellow!20, thick}
    \sbStyleSum{minimum size=1em, fill=black!50, thick}
    \sbEntree{input}
    \sbComp*[4]{comp}{input}
        \sbRelier{input}{comp}
        \sbNomLien[0.8]{input}{$V_a(t)$}
    \sbBloc[3]{system}{$\dfrac{K}{1+Ts}$}{comp}
        \sbRelier[$e(t)$]{comp}{system}
    \sbSortie[4]{output}{system}
        \sbRelier{system}{output}
        \sbNomLien[0.8]{output}{$\omega(t)$}
    \sbRenvoi{system-output}{comp}{}
        \fill (system-output.south) circle[radius=1pt];
    
\end{tikzpicture}


\subsection{Position feedback control}

Figure~\ref{fig:PositionFeedback} shows the block diagram for the
position feedback servo system.

%
\tikzsetnextfilename{Position_Feedback}
\begin{tikzpicture}[every node/.style={font=\small}]
    \sbStyleSum{minimum size=1em, fill=black!50, thick}
    \sbEntree{input}
    \sbComp*[4]{comp}{input}
        \sbRelier{input}{comp}
        \sbNomLien[0.8]{input}{$\theta_i(t)$}
    \sbStyleBloc{minimum width=3em, fill=red!20, thick}
    \sbBloc[3]{gain}{$K_a$}{comp}
        \sbRelier[$e(t)$]{comp}{gain}
    \sbStyleBloc{minimum width=3em, fill=yellow!20, thick}
    \sbBloc[3]{system}{$\dfrac{K}{s(1+Ts)}$}{gain}
        \sbRelier{gain}{system}
    \sbSortie[4]{output}{system}
        \sbRelier{system}{output}
        \sbNomLien[0.8]{output}{$\theta_o(t)$}
    \sbRenvoi{system-output}{comp}{}
        \fill (system-output.south) circle[radius=1pt];
    
\end{tikzpicture}


\subsection{Velocity feedback control}

Figure~\ref{fig:VelocityFeedback} shows the block diagram for the
position feedback servo system.

%
\tikzsetnextfilename{Velocity_Feedback}
\begin{tikzpicture}[every node/.style={font=\small}]
    \sbStyleSum{minimum size=1em, fill=black!50, thick}
    \sbEntree{input}
    \sbCompSum*[4]{comp}{input}{-}{-}{+}{}
        \sbRelier{input}{comp}
        \sbNomLien[0.8]{input}{$\theta_i(t)$}
    \sbStyleBloc{minimum width=3em, fill=red!20, thick}
    \sbBloc[3]{gain}{$K_a$}{comp}
        \sbRelier[$e(t)$]{comp}{gain}
    \sbStyleBloc{minimum width=3em, fill=yellow!20, thick}
    \sbBloc[3]{system}{$\dfrac{K}{(1+Ts)}$}{gain}
        \sbRelier{gain}{system}
    \sbBloc[3]{integral}{$\dfrac{1}{s}$}{system}
        \sbRelier{system}{integral}
    \sbSortie[4]{output}{integral}
        \sbRelier{integral}{output}
        \sbNomLien[0.8]{output}{$\theta_o(t)$}
    \sbRenvoi{integral-output}{comp}{}
        \fill (integral-output.south) circle[radius=1pt];
    \sbDecaleNoeudy[-4]{system}{VfRef}
    \sbStyleBloc{minimum width=3em, fill=red!20, thick}
    \sbBlocr[3]{Vf}{$K_f$}{VfRef}
    \sbRelieryx{system-integral}{Vf}
        \fill (system-integral.south) circle[radius=1pt];
    \sbRelierxy{Vf}{comp}
    
\end{tikzpicture}


\subsection{Velocity control system}

Figure~\ref{fig:VelocityControl} shows the block diagram for the
speed controlled system.

%
\tikzsetnextfilename{Velocity_Control}
\begin{tikzpicture}[every node/.style={font=\small}]
    \sbStyleSum{minimum size=1em, fill=black!50, thick}
    \sbEntree{input}
    \sbCompSum*[4]{comp}{input}{-}{}{+}{}
        \sbRelier{input}{comp}
        \sbNomLien[0.8]{input}{$\omega_i(t)$}
    \sbStyleBloc{minimum width=3em, fill=red!20, thick}
    \sbBloc[3]{gain}{$K_a$}{comp}
        \sbRelier[$e(t)$]{comp}{gain}
    \sbStyleBloc{minimum width=3em, fill=yellow!20, thick}
    \sbBloc[3]{system}{$\dfrac{K}{(1+Ts)}$}{gain}
        \sbRelier{gain}{system}
    \sbSortie[4]{output}{system}
        \sbRelier{system}{output}
        \sbNomLien[0.8]{output}{$\omega_o(t)$}
    \sbDecaleNoeudy[-4]{system}{VfRef}
    \sbStyleBloc{minimum width=3em, fill=red!20, thick}
    \sbBlocr[3]{Vf}{$K_f$}{VfRef}
    \sbRelieryx{system-output}{Vf}
        \fill (system-output.south) circle[radius=1pt];
    \sbRelierxy{Vf}{comp}
    
\end{tikzpicture}

%
\tikzsetnextfilename{PIVelocity}
\begin{tikzpicture}[every node/.style={font=\small}]
    \sbStyleSum{minimum size=1em, fill=black!50, thick}
    \sbEntree{input}
    \sbCompSum*[4]{comp}{input}{-}{-}{+}{}
        \sbRelier{input}{comp}
        \sbNomLien[0.8]{input}{$\theta_i(t)$}
    \sbStyleBloc{minimum width=3em, fill=red!20, thick}
    \sbDecaleNoeudx[1.5]{comp}{piref}
    \sbDecaleNoeudy[-3]{piref}{pref}
    \sbDecaleNoeudy[3]{piref}{iref}
    \sbBloc[1.5]{p}{$1$}{pref}
    \sbBloc[1.5]{i}{$\dfrac{1}{T_i s}$}{iref}
    \sbCompSum*[6]{pisum}{piref}{+}{+}{}{}
        \filldraw (comp.east) -- (piref.center) circle[radius=1pt];
%        \sbRelier{comp}{piref}
        \sbRelieryx{piref}{p}
        \sbRelierxy{p}{pisum}
        \sbRelieryx{piref}{i}
        \sbRelierxy{i}{pisum}
    \sbBloc[2]{gain}{$K_a$}{pisum}
        \sbRelier{pisum}{gain}
    \sbStyleBloc{minimum width=3em, fill=yellow!20, thick}
    \sbBloc[2]{system}{$\dfrac{K}{(1+Ts)}$}{gain}
        \sbRelier{gain}{system}
    \sbBloc[2]{integral}{$\dfrac{1}{s}$}{system}
        \sbRelier{system}{integral}
    \sbSortie[4]{output}{integral}
        \sbRelier{integral}{output}
        \sbNomLien[0.8]{output}{$\theta_o(t)$}
    \sbRenvoi[7]{integral-output}{comp}{}
        \fill (integral-output.south) circle[radius=1pt];
    \sbDecaleNoeudy[-7]{system}{VfRef}
    \sbStyleBloc{minimum width=3em, fill=red!20, thick}
    \sbBlocr[2]{Vf}{$K_f$}{VfRef}
    \sbRelieryx{system-integral}{Vf}
        \fill (system-integral.south) circle[radius=1pt];
    \sbRelierxy{Vf}{comp}
    
\end{tikzpicture}


%
\tikzsetnextfilename{PID0}
\begin{tikzpicture}[every node/.style={font=\small}]
    \sbStyleSum{minimum size=1em, fill=black!50, thick}
    \sbEntree{input}
    \sbComp*[4]{comp}{input}
        \sbRelier{input}{comp}
        \sbNomLien[0.8]{input}{$\theta_i(t)$}
    \sbStyleBloc{minimum width=3em, fill=red!20, thick}
    \sbBloc[5]{p}{$1$}{comp}
    \sbDecaleNoeudy[-4]{p}{iref}
    \sbDecaleNoeudy[4]{p}{dref}
    \sbBloc[-1.5]{i}{$\dfrac{1}{T_i s}$}{iref}
    \sbBloc[-1.5]{d}{$T_d s$}{dref}
    \sbCompSum*[4]{pidsum}{p}{+}{+}{+}{}
        \sbRelier{comp}{p}
        \sbRelier{p}{pidsum}
        \sbRelieryx{$(comp.east)!.6!(p.west)$}{i}
        \sbRelierxy{i}{pidsum}
        \sbRelieryx{$(comp.east)!.6!(p.west)$}{d}
        \sbRelierxy{d}{pidsum}
        \fill ($(comp.east)!.6!(p.west)$) circle[radius=1pt];
%        \sbNomLien{$(comp.east)!.3!(p.west)$}{$e(t)$}
        \node[above] at ($(comp.east)!.3!(p.west)$) {$e(t)$};
    \sbBloc[2]{gain}{$K_a$}{pidsum}
        \sbRelier{pidsum}{gain}
    \sbStyleBloc{minimum width=3em, fill=yellow!20, thick}
    \sbBloc[3]{system}{$\dfrac{K}{s(1+Ts)}$}{gain}
        \sbRelier{gain}{system}
    \sbSortie[4]{output}{system}
        \sbRelier{system}{output}
        \sbNomLien[0.8]{output}{$\theta_o(t)$}
    \sbRenvoi[7]{system-output}{comp}{}
        \fill (system-output.south) circle[radius=1pt];

\end{tikzpicture}


%
\tikzsetnextfilename{PID1}
\begin{tikzpicture}[every node/.style={font=\small}]
    \sbStyleSum{minimum size=1em, fill=black!50, thick}
    \sbEntree{input}
    \sbComp*[4]{comp}{input}
        \sbRelier{input}{comp}
        \sbNomLien[0.8]{input}{$\theta_i(t)$}
    \sbStyleBloc{minimum width=3em, fill=red!20, thick}
    \sbBloc[5]{p}{$1$}{comp}
    \sbDecaleNoeudy[-4]{p}{iref}
    \sbDecaleNoeudy[4]{p}{dref}
    \sbBloc[-1.5]{i}{$\dfrac{1}{T_i s}$}{iref}
    \sbBloc[-1.5]{d}{$T_d s$}{dref}
    \sbCompSum*[4]{pidsum}{p}{+}{-}{+}{}
        \sbRelier{comp}{p}
        \sbRelier{p}{pidsum}
        \coordinate (pidref) at ($(comp.east)!.6!(p.west)$);
%        \sbDecaleNoeudx[3.5]{comp}{pidref}
        \sbRelieryx{pidref}{i}
        \sbRelierxy{i}{pidsum}
%        \sbRelieryx{$(comp.east)!.6!(p.west)$}{d}
        \sbRelierxy{d}{pidsum}
        \fill (pidref) circle[radius=1pt];
%        \sbNomLien{$(comp.east)!.3!(p.west)$}{$e(t)$}
        \node[above] at ($(comp.east)!.5!(pidref.west)$) {$e(t)$};
    \sbBloc[2]{gain}{$K_a$}{pidsum}
        \sbRelier{pidsum}{gain}
    \sbStyleBloc{minimum width=3em, fill=yellow!20, thick}
    \sbBloc[3]{system}{$\dfrac{K}{s(1+Ts)}$}{gain}
        \sbRelier{gain}{system}
    \sbSortie[4]{output}{system}
        \sbRelier{system}{output}
        \sbNomLien[0.8]{output}{$\theta_o(t)$}
    \sbRenvoi[7]{system-output}{comp}{}
        \fill (system-output.south) circle[radius=1pt];
    \coordinate[below=7em of pidref] (dref);
    \fill (dref) circle[radius=1pt];
%    \sbDecaleNoeudy[7]{pidref}{dref}
        \sbRelieryx{dref}{d}
%    \sbDecaleNoeudy[-7]{system}{VfRef}
%    \sbStyleBloc{minimum width=3em, fill=red!20, thick}
%    \sbBlocr[2]{Vf}{$K_f$}{VfRef}
%    \sbRelieryx{system-integral}{Vf}
%    \sbRelierxy{Vf}{comp}
    
\end{tikzpicture}


%
\tikzsetnextfilename{PID2}
\begin{tikzpicture}[every node/.style={font=\small}]
    \sbStyleSum{minimum size=1em, fill=black!50, thick}
    \sbEntree{input}
    \sbComp*[4]{comp}{input}
        \sbRelier{input}{comp}
        \sbNomLien[0.8]{input}{$\theta_i(t)$}
    \sbStyleBloc{minimum width=3em, fill=red!20, thick}
    \sbBloc[5]{p}{$1$}{comp}
    \sbDecaleNoeudy[-4]{p}{iref}
    \sbDecaleNoeudy[4]{p}{dref}
    \sbBloc[-1.5]{i}{$\dfrac{1}{T_i s}$}{iref}
%    \sbBloc[-1.5]{d}{$T_d s$}{dref}
    \sbCompSum*[4]{pidsum}{p}{+}{}{+}{}
        \sbRelier{comp}{p}
        \sbRelier{p}{pidsum}
        \coordinate (pidref) at ($(comp.east)!.6!(p.west)$);
%        \sbDecaleNoeudx[3.5]{comp}{pidref}
        \sbRelieryx{pidref}{i}
        \sbRelierxy{i}{pidsum}
%        \sbRelieryx{$(comp.east)!.6!(p.west)$}{d}
%        \sbRelierxy{d}{pidsum}
        \fill (pidref) circle[radius=1pt];
%        \sbNomLien{$(comp.east)!.3!(p.west)$}{$e(t)$}
        \node[above] at ($(comp.east)!.5!(pidref.west)$) {$e(t)$};
	\sbCompSum*[3]{pidsum1}{pidsum}{}{-}{+}{}
		\sbRelier{pidsum}{pidsum1}
    \sbBloc[1.5]{gain}{$K_a$}{pidsum1}
        \sbRelier{pidsum1}{gain}
    \sbStyleBloc{minimum width=3em, fill=yellow!20, thick}
    \sbBloc[2]{system}{$\dfrac{K}{s(1+Ts)}$}{gain}
        \sbRelier{gain}{system}
    \sbSortie[4]{output}{system}
        \sbRelier{system}{output}
        \sbNomLien[0.8]{output}{$\theta_o(t)$}
    \sbRenvoi[7]{system-output}{comp}{}
        \fill (system-output.south) circle[radius=1pt];
	\sbStyleBloc{minimum width=3em, fill=red!20, thick}
	\sbDecaleNoeudy[4]{system}{dref}
	\sbBlocr[2]{d}{$T_d s$}{dref}
		\coordinate (dfbref) at ($(system.east)!.25!(output.west)$);
		\fill (dfbref) circle[radius=1pt];
		\sbRelieryx{dfbref}{d}
		\sbRelierxy{d}{pidsum1}
    
\end{tikzpicture}

%
% Define the layers for the diagram
\pgfdeclarelayer{background}
\pgfdeclarelayer{foreground}
\pgfsetlayers{background,main,foreground}

\tikzsetnextfilename{PIDBode}
\begin{tikzpicture}[>=latex', 
    ref lines/.style={thin, blue!60}, 
    ref points/.style={circle, black, opacity=0.7, fill, minimum size= 3pt, inner sep=0}, 
    every node/.style={font=\small}, 
    bode lines/.style={very thick, blue}, 
    Gclabel/.style={text=blue}, 
    xscale=12/3]
    
\begin{scope}[yscale=4/110]
\UnitedB
\semilog{-1}{2}{-50}{60}

% Bode plot (magnitude) for the original system, 4/(s/(1+2s)). 
% Asymptotic line
\BodeAmp[ref lines, red!60]{-1:1.35}{\POAmpAsymp{4}{2.0}+\IntAmp{1}}
% Bode plot
\BodeAmp[bode lines, red, name path=Gomagnitude]{-1:1.35}{\POAmp{4}{2.0}+\IntAmp{1}}

% Bode plot (magnitude) for the original system, 4(1+s)/(s^2/(1+0.1s)). 
% Asymptotic line
\BodeAmp[ref lines]{-1:2}{\APAmpAsymp{4}{0.1}{10}+2*\IntAmp{1}}
% Bode plot
\BodeAmp[bode lines, name path=magnitude]{-1:2}{\APAmp{4}{0.1}{10}+2*\IntAmp{1}}

% Reference line, (0dB)
\draw [name path=unitygain, ref lines] (-1,0) -- (2,0);

% Crossover frequency of the original system
\path [name intersections={of=magnitude and unitygain, by=crossover}];
\node (coref) [ref points, label={[blue]below left:$\omega_c$}] at (crossover) {};

% Crossover frequency of the compensated system
\path [name intersections={of=Gomagnitude and unitygain, by=Gocrossover}];
\node (Gocoref) [ref points, label={[red]below:$\omega_{co}$}] at (Gocrossover) {};

% Labels for the original system (open-loop)
\node [Gclabel, text=red] at (-0.7, 15) {$-20$dB/dec};
\node [Gclabel, text=red] at (0.5, -30) {$-40$dB/dec};
\node (KG) [Gclabel, red!60, text=red, draw] at (-0.5, -30) {$KG(s)=\dfrac{4}{s(1+2s)}$};
\draw (KG.east) edge [->, shorten >=1pt, thick, red, bend right=1.5] (0.4, -10);

% Labels for the compensated system (open-loop)
\node [ref points, label={[blue]below left:$\omega_2$}] at (0, 0) {};
\node [ref points, label={[blue]below right:$\omega_3$}] at (1, 0) {};
\node [Gclabel] at (-0.4, 40) {$-40$dB/dec};
\node [Gclabel] at (0.5, 10) {$-20$dB/dec};
\node [Gclabel] at (1.6, -20) {$-40$dB/dec};
\node (KDG) [Gclabel, blue!60, text=blue, draw] at (1.4, 40) {$KD(s)G(s)=\dfrac{4(1+s)}{s^2(1+0.1s)}$};
\draw (KDG.west) edge [->, shorten >=1pt, thick, blue, bend right=1.5] (0.17, 10);

% Axes label
\node [below=6pt] at (0.5,-50) {Frequency, $\omega$};
\node [rotate=90] at (-1.25,5) {Magnitude, $20\log(|G(\text{j}\omega)|)$};


\end{scope}

\begin{scope}[yshift=-3.5cm,yscale=4/180]
\UniteDegre
\OrdBode{30}
\semilog{-1}{2}{-180}{0}
% Bode plot (phase) for the original system, 4/(s/(1+2s)). 
% Asymptotic line
\BodeArg[ref lines, red!60]{-1:2}{\POArgAsymp{4}{2.0}+\IntArg{1}}
% Bode plot
\BodeArg[bode lines, red, name path=Gophase]{-1:2}{\POArg{4}{2}+\IntArg{1}}

% Bode plot (magnitude) for the original system, 4(1+s)/(s^2/(1+0.1s)). 
% Asymptotic line
\BodeArg[ref lines]{-1:2}{\APArgAsymp{4}{0.1}{10}+2*\IntArg{1}}
% Bode plot
\BodeArg[bode lines, name path=phase]{-1:2}{\APArg{4}{0.1}{10}+2*\IntArg{1}}

% Phase margin of the original system
\path [name path=Gowcref] (Gocrossover) -- ++(0,-330);
\path [name intersections={of=Gophase and Gowcref, by=Gophasemargin}];
\node (Gopmref) [ref points] at (Gophasemargin) {};
\draw [ref lines, red!60, densely dotted] (Gocoref.south) -- (Gopmref.north);
\draw [ref lines, <->, red] let \p1=(Gophasemargin)
    in
        (Gopmref.south) -- node[left, Gclabel, text=red] {$\text{PM}_o$} (\x1,-180);
        
% Phase margin of the compensated system
\path[name path=wcref] (crossover) -- ++(0,-300);
\path [name intersections={of=phase and wcref, by=phasemargin}];
\node (pmref) [ref points] at (phasemargin) {};
\draw [ref lines, densely dotted] (coref.south) -- (pmref.north);
\draw [ref lines, <->, blue] let \p1=(phasemargin)
    in
        (pmref.south) -- node[left, Gclabel] {PM} (\x1,-180);

       
\node (KGphase) [Gclabel, red!60, text=red, draw] at (-0.5, -40) {$KG(s)=\dfrac{4}{s(1+2s)}$};
\draw[->, shorten >=1pt, thick, red] (KGphase.south) .. controls +(down:30) and +(0.1,10) .. (-0.65, -114);

\node (KDGphase) [Gclabel, blue!60, text=blue, draw] at (1.4, -40) {$KD(s)G(s)=\dfrac{4(1+s)}{s^2(1+0.1s)}$};
\draw[->, shorten >=1pt, thick, blue] (KDGphase.south) .. controls +(down:40) and +(0.1,30) .. (1.1, -146);

% Axes label        
\node [below=6pt] at (0.5, -180) {Frequency, $\omega$};
\node [rotate=90] at (-1.25, -90) {Phase, $\angle G(\text{j}\omega)$};
\end{scope}

% Background, try tkzexample later.
\begin{pgfonlayer}{background}
    \path (-1.4cm,2.8cm) node (tl) {};
    \path (2.3cm, -8.4cm) node (br) {};
    
    \path[fill=brown!20] (tl) rectangle (br);
\end{pgfonlayer}

\end{tikzpicture}


\section{Conclusions}


\section*{Tmp}

A bug of asmmath, can be fixed by mathtools

\[
\begin{gathered}{}[p]=100\\
{}[v]=200\end{gathered}
\]

\end{document}
